\documentclass[12pt]{article}
\usepackage{amsmath}
\usepackage{graphicx}
\usepackage[american]{babel}

\usepackage[T1]{fontenc}
\usepackage[utf8]{inputenc}
\usepackage{indentfirst}

\usepackage{csquotes}
\usepackage[authordate,backend=biber,url=true]{biblatex-chicago}
\addbibresource{references.bib}

% Load hyperref late to ensure all anchors are created
\usepackage[colorlinks=true,linkcolor=blue,citecolor=blue,urlcolor=blue]{hyperref}


% Margins
\usepackage[margin=1in]{geometry}

% Double spacing
\usepackage{setspace}
\doublespacing

% Optional: paragraph formatting
\setlength{\parindent}{0.5in} % classic Word-style indent
\setlength{\parskip}{0pt}     % no extra space between paragraphs

\makeatletter

\renewcommand*\singlespacing{%
    \par    % ensure vertical mode
    \null   % add fake line with previous leading still in force
    \setstretch {\setspace@singlespace}% change leading
    \nobreak
    \vskip -\baselineskip   % compensate for the fake line we added, but with 
                            % the new leading
    \vskip \z@skip  % tell "\addvspace" and "\addpenalty" _not_ to remove the 
                    % above correction
}

\makeatother


\begin{document}
\begin{titlepage}
    \centering
    
    % push content down ~1/3 of the page
    \vspace*{0.2\textheight}
    
    {\Large NYC Congestion Pricing:}\\[0.5em]
    {\Large Honors Thesis Research Proposal}
    \emph{v2.0}\\[4em]
    
    {\large Alejandro Haerter}\\[2em]
    
    {\large ECON 4494W -- ECON 4497W}\\[0.5em]
    {\large Spring 2025 -- December 2026}\\[2em]
    
    {\large University of Connecticut}\\
    
    \vfill
\end{titlepage}

\subsection*{Motivation and Research Question}
Traffic congestion is one of the most persistent challenges facing U.S. metropolitan
areas, with wide-ranging social, economic, and environmental costs. It reduces
quality of life, slows economic productivity, and contributes to pollution.
From an economic perspective, the problem is straightforward: time and money lost
in traffic are scarce resources that could be more productively allocated. For decades,
economists have been searching for an optimal solution to congestion.

Among urban areas, the New York City metro area---particularly encompassing the city's 
five boroughs and its constituent suburban counties (e.g., Westchester County NY,
Fairfield County CT)---provides a uniquely diverse landscape for analyzing congestion.
The NYC metro area's transportation network garners the attention of the federal government,
three state governments, and several regional agencies. 
This results in an interesting mix of transportation policies, including toll-free highways, 
traditional toll roads,
public mass transit, and Manhattan's recently implemented congestion pricing.
Historically, policymakers---often at the behest of advocacy groups---have approached the
traffic congestion problem by expanding roadway and/or public transportation capacity.
Unfortunately for this approach, the extant economics literature increasingly questions
the efficacy of these traditional policy responses, noting little to no long-term decreases
in congestion. While initially appearing unintuitive, multiple recent studies suggest
that capacity expansions induce demand onto roadways by lowering travel costs, resulting
in no net change in congestion. Public transit expansions often fail to significantly
reduce congestion for this same reason; any substitution effect is simply negated by
induced demand. These realizations have spurred transportation economists in a new direction:
congestion pricing. Unfortunately, economists have only had theory to work with, because
until 2025, there was no congestion pricing in the U.S.

On January 5th, 2025, the borough of Manhattan introduced congestion pricing in an attempt
to curb the infamous New York City traffic while also serving as a form of funding for the
NYC public transportation system. 
Given the economic theory, the diverse transportation network, and the new congestion pricing,
I seek to answer: Does congestion pricing reduce highway congestion more effectively than
road expansion and transit investment in the NYC metro area? If so, by how much?

I propose the first empirical analysis of the efficacy congestion pricing in the U.S.
This study would evaluate different government solutions with the goal to substantiate
the claims made by transportation economic theory, where congestion pricing is considered
the truly pragmatic optimal solution. I would have to compare
infrastructure investments, toll-based strategies, and congestion pricing policies.

\subsection*{Literature Review}
The extant literature on traffic congestion comprehensively outlines the limitations of
simple capacity expansions. Increasingly, especially over the last decade, the literature
has been more interested in exploring congestion pricing as an optimal solution to road
congestion.

\textcite{Winston2006} measures the effects of highway spending on road users' congestion
costs. Congestion costs are the economic losses caused by road traffic congestion. 
They measure the difference between travel in free-flow conditions and travel under congested
conditions. The study's results display dramatic inefficiency: an eleven cents reduction
in road users' congestion costs per dollar of government highway spending
\parencite[480]{Winston2006}.

\begin{figure}[t]
  \centering
  \includegraphics[width=0.8\textwidth]{road_sd.png}
  \caption{Supply and Demand for Road Traffic. Source: \textcite{Duranton2011}.}
  \label{fig:mkt1}
\end{figure}

Critical work by \textcite{Duranton2011} expands on the disappointing result from
\textcite{Winston2006} by offering a nation-wide econometric analysis.
\textcite{Duranton2011} describes "[the] fundamental law
of highway congestion: people drive more when the stock of roads in their city increases;
commercial driving and trucking increase with a city's stock of roads; and people migrate
to cities that are relatively well provided with road" (2618). This problem is illustrated by
Figure \ref{fig:mkt1}, where $Q$ is vehicle kilometers traveled (VKT),
$P(Q)$ is travel cost, and $R$ is the supply of roads. At equilibrium,
$P(Q)$ is equal to supply curve $AC(R)$. So, an increase in road supply from $R$ to
$R'$ moves the supply curve to $AC(R')$, thus decreasing travel cost, and proportionally
inducing demand. There is no long-term congestion relief in this framework. 

Unfortunately, two issues arise when considering mass transit as a congestion solution.
First, the same induced demand principle applies for public transportation expansions;
if a public transit expansion removes a subset of drivers from the roads, a new subset
of drivers would simply take their place, returning to a congestion equilibrium
\parencite[2636]{Duranton2011}. 
Second, these expansions may not alter existing commuter behavior sufficiently
in the first place. Short-run evidence from Chinese cities demonstrates that
subway-induced congestion relief is strongest in close proximity to subway-stops, and
only moves drivers away from already congested arterial roads \parencite[17]{Gu2021}.
This is not good news for New York City mass transit investment. In the New York City
suburbs, which are serious contributors to the city's congestion, mass transit is
significantly more sparse. These areas are unlikely to benefit from congestion relief
that capacity improvements could provide. For example, the ubiquitous New York City
Subway primarily serves intra-city commuters; we would not expect subway capacity improvements
to relieve congestion on surface roads comprising of commuters coming to and from the
suburbs.

\begin{figure}[t]
  \centering
  \includegraphics[width=0.8\textwidth]{road_sd2.png}
  \caption{Supply and Demand for Road Traffic with Toll. Adapted from: \textcite{Duranton2011}.}
  \label{fig:mkt2}
\end{figure}

With these discoveries in mind, I ought to conduct my research with serious scrutinization
of road and mass transit capacity expansions as efficacious traffic solutions, and thus
direct my attention to a very different solution: toll-based
strategies. Tolls are deceptively simple; they may appear to only serve as revenue collection
but are fundamentally able to reduce congestion. Recall the example from Figure \ref{fig:mkt1}.
In this same example, imposing a roadway toll of  $t$ per-kilometer makes the average cost of
travel more expensive, shifting the supply curve to $AC(R)+t$. A new equilibrium is found
at a reduced $Q^{*t}$: the toll reduces traffic. This scenario is reflected in
Figure \ref{fig:mkt2}. Most traditional toll systems, e.g., the E-ZPass system found in 
New York and New Jersey, operate this way.

Of course, this toll behaves like any Pigouvian per-unit 
tax\footnote{See \textcite{Bento2009} for an exploration of congestion relief
via gasoline tax, which theoretically behaves the same.},
which unfortunately results in market inefficiency.
By definition this is a suboptimal solution to traffic congestion, and this is where
economists turn their attention to congestion pricing. \textcite{Winston2013} concludes:
\begin{quote}
    \singlespacing
    Vehicles should be charged for their use of lane capacity that contributes to
    congestion by paying efficient (marginal congestion) congestion tolls\ldots
    out-of-pocket cost of commuting would no longer be underpriced, such tolls could
    generate annual gains of \$40 billion, account for the travel time savings for
    commuters, savings for taxpayers from lower costs of public services from greater
    residential density, and greater revenues to the government. 
    (787)
\end{quote}

Theoretically, a perfect congestion-pricing scheme charges an amount exactly equal
to the marginal social cost (MSC) of a vehicle's trip, e.g., the contribution to 
congestion, emissions, etc. That way, a socially efficient level
of traffic is achieved. Of course, this is infeasible, and a second-best practical
alternative involves only tolling some roads, using flat-rates, and revenue
recycling \parencite{Parry2009}.
The automated transponder-based toll technology available in
2025 means this congestion toll would not require any additional infrastructure beyond a
traditional toll, allowing relative ease of implementation.

That being said, infrastructure installation is not the only obstacle to implementation.
Political and equity considerations often pose
significant implementation challenges; lower-income demographics may be disproportionately
affected by congestion charges as they do not account for income. Additional, environmental
benefits are modest at best. Public opinion is also a concern; worries over revenue use and
perceived unfairness can delay or halt implementation \parencite{Anas2011,Lindsey2012}.

The extant literature thus provides a clear answer as to why congestion pricing should be the
primary interest of this research, but I need to caution for the potential drawbacks of congestion
pricing as well and include these in my results.

\subsection*{Data and Source Material}

To quantitatively evaluate the impact of transportation spending policies on highway
congestion, I will rely mostly on publicly available datasets. These will include
traffic volume data from sources under the U.S. Department of Transportation's (USDOT) umbrella.
Some examples include the Federal Highway Administration (FHWA), the Port Authority
of New York and New Jersey (PANYNJ), the Metropolitan Transportation Authority (MTA),
and of course state DOTs (NYSDOT, CTDOT, NJDOT). As mentioned in the introduction, the
NYC metro area lies in three states, which means there are several government agencies---
each with their own interests---involved. These agencies do not necessarily cooperate, and so
a policy enacted in one state does not mean it will be enacted in another. An example is that
highways in New York and New Jersey impose tolls, but tolls do not exist anywhere in Connecticut,
despite also suffering from relatively high congestion caused by the commute to NYC.

Road use/congestion data can be difficult as it can originate from multiple sources. While interstate
highway data is relatively easy to obtain (directly from USDOT), smaller roads become more difficult.
From my brief search, it appears road use datasets may originate at the municipal level. This
could potentially make the data scope too large; upon beginning research I will have to decide if
I need to narrow the scope of road use to interstate highway as done in \textcite{Duranton2011}.

Congestion pricing is currently only in effect in the Congestion Relief Zone (CRZ), which
is delineated by all Manhattan roads south of 60th street (i.e., everything south of Central Park),
excluding the West Side Highway and FDR Drive.
Toll-specific data regarding infrastructure, pricing, etc., will be obtained from the MTA.
The MTA manages the congestion pricing network and several of the tolls for bridges and tunnels
in New York City. Outside of NYC, tolls are managed by the state DOTs or PANYNJ. 

Public transportation data, if required, comes from the MTA, CTDOT, or NJDOT, for their respective
states. Ridership numbers are not often made publicly available besides an annual aggregate;
if I decide to include public transport analysis in my research I may have to employ FOIA requests.

Previous econometric literature \parencite{Duranton2011,Gu2021,Barwick2024} makes rigorous
use of economic and demographic variables in their modeling, essential for controlling all of
the confounding effects found in a cityscape. The U.S. Census Bureau's American Community Survey
(ACS) is a good source for these. Variables such as population density, median household income,
commuter modal choices, and commuting times come from other surveys that the U.S. Census conducts.

Evidently, I am met with an abundance of datasets, which I will necessarily need to narrow to a
feasible scope of research. Pooling these cross-sections may be incredibly difficult, especially
without some sort of robust population control (likely nonexistent), so I will have to account
for this difficulty when modeling.

Finally, supplementary qualitative sources, including policy evaluations, governmental reports,
and probably local planning documents, will provide context and insight into implementation details
and political considerations influencing policy effectiveness. 


\subsection*{Methodology}
An econometric approach is necessary to evaluate these government solutions. This analysis
lends itself to panel data regression modeling for a simpler approach; this may
allow me to establish causal inference for the impacts of different policies on congestion
levels. A large data panel should allow me to account for variations over time and in location,
as well as to assist in controlling for unobserved heterogeneity.

One preliminary econometric model might take the form of a differences-in-differences
(DiD)\footnote{
See \textcite{Gu2021} for an implementation of a stacked DiD model to measure congestion response,
useful for when different cities enact the treatment at different times. If another U.S. city
were to enact congestion pricing, we might use this model.} design,
to compare the before-and-after of NYC congestion. In this model, both spatial and temporal
variation are controlled for. Consider a simple two-way fixed-effects model: \singlespacing
\begin{equation}
    \text{Congestion}_{it} = \beta + \delta(Post_{t} \times Treat_{i})+ \mu_{i} + \lambda_{t} + \epsilon_{it},
\end{equation}
where:
\begin{align*}
    \text{Congestion}_{it} &= \text{some measure of traffic (average travel speed, VKT, throughput)}\\
    \beta &= \text{intercept coefficient}\\
    \delta &= \text{DiD coefficient}\\
    \mu_i &= \text{spatial fixed effects}\\
    \lambda_t &= \text{temporal fixed effects}\\
    \epsilon_{it} &= \text{random error}.
\end{align*}
\doublespacing A significant DiD estimator $\delta$ means that the treated units (e.g., Manhattan) 
had a change in congestion after policy implementation that \emph{was not} reflected in the controls.
This model requires a nuanced variable selection. To begin with, the observation $i$
depends on the choice of control variables. One comparison would be the CRZ against other places in the
NYC metro area, including uptown Manhattan, other boroughs, and adjacent suburban towns. This approach
could showcase a decrease in congestion in the CRZ that other areas do not, but this assumes
that these controls will \emph{not} respond to congestion pricing at all. In reality, a significant portion
of NYC area traffic is destined for the CRZ, and nearby boroughs and feeder towns may too experience
a decrease in congestion. In this case, using surrounding areas as controls risks underestimating the true
effect of congestion pricing, since the treatment indirectly benefits the controls.

Thus, a larger model may be necessary, instead considering controls from areas outside of the NYC
metro area. Areas from large cities with congestion issues, e.g., Chicago, Los Angeles, Atlanta, would
serve as controls entirely unaffected by the CRZ. However, this framework introduces its own challenges.
The central assumption in a DiD framework is parallel trends: that absent treatment, congestion in treated
and control units would have evolved similarly. While other large metros share structural similarities 
with New York—dense cores, high vehicle demand, multimodal transport systems—the evolution of their 
congestion may differ due to local policy changes, infrastructure investment, or regional economic shocks.
Therefore, including outside cities provides cleaner isolation from spillovers but raises questions about
the validity of the parallel trends assumption. Pre-trend checks for parallelism would be necessary.

A reasonable compromise is a hybrid approach: the primary specification compares Manhattan with nearby boroughs
and suburban counties, while a secondary specification incorporates other large U.S. cities as robustness checks.
If both designs yield consistent estimates of the CRZ’s effect, confidence in the results is strengthened; 
if they diverge, the comparison helps identify whether spillovers or non-parallel trends drive the discrepancy.

Provided I establish a relationship between congestion pricing and traffic, it is an
interesting exercise to evaluate the extent of this impact. My intuition is that
congestion pricing is unlikely to affect all surrounding areas equally: towns and boroughs closer to the CRZ
are more directly linked to the Manhattan commuting flows and thus more likely to exhibit measurable changes,
while distant suburbs may remain unaffected. To capture this gradient, the model can be augmented with an
interaction between the post-treatment indicator and a continuous measure of distance from the CRZ:
\begin{equation}
    \text{Congestion}_{it} = \beta + \delta(Post_t \times Treat_i) + \theta(Post_t \times Distance_i) + \mu_i
    + \lambda_t + \epsilon_{it},
\end{equation}
where $\theta$ captures how much the treatment effect decays as distance from Manhattan increases.
A negative and significant $\theta$ would imply that congestion relief is concentrated near the CRZ and
diminishes with distance, consistent with the intuition of a “spatial spillover gradient.” This
specification recognizes that spillovers are not binary (treated vs. control) but continuous,
and allows the model to account for the geography of commuting patterns across the NYC metropolitan area.

Finally, an additional extension to this model would test a frequent criticism of congestion pricing,
its disproportionate distributional/equity impacts \parencite{Barwick2024,Parry2009,Lindsey2012}.
Consider:
\begin{equation}
    \text{Congestion}_{it} = \beta + \delta(Post_t \times Treat_i) + \phi(Post_t \times Wealth_i) + \mu_i
    + \lambda_t + \epsilon_{it},
\end{equation}
The interaction term on $\phi$ interacts treatment with area-level income to test whether wealthier and
poorer communities respond differently to congestion pricing. If theory holds, wealthier communities should
be able to eat the toll (and thus, have less impact). Confounding is a potential issue here and may have
to require the addition of the interaction term found in the distance model, if community wealth is somehow
attributed to distance from the CRZ.

It should be stated that (1), (2), and (3) are each preliminary models which will likely have additional
complexity once the scope of the data is determined and overview statistics are computed. It is unlikely
that I will be able to use so simple of a model for a robust result; but at the same time, more complex
models require more data, and with congestion pricing only being introduced in 2025, this poses a
significant challenge. The final model must find this balance, which I anticipate to be one of the
most significant obstacles.

Beyond the basic DiD, structural models are able to capture mechanisms, and simulate counterfactuals.
\textcite{Brinkman2016} utilizes a structural spatial equilibrium model which
simulates the effects of congestion pricing on Columbus, Ohio. This approach may be replicated
to compare a simulation of his approach to the actual empirical effect observed by an econometric
model on NYC. If the \textcite{Brinkman2016} model holds, it can confidently be used on other U.S.
cities. What makes this model interesting is that it accounts for productivity losses which may
be caused by agglomeration loss; congestion pricing disperses jobs away from concentrated districts
affected by the toll. In Manhattan, where finance services cluster, this productivity loss risk
is even greater than in Columbus. Testing Brinkman’s framework against NYC data could reveal whether
the welfare ambiguity he predicts holds in larger, more clustered cities.

\textcite{Barwick2024}'s equilibrium sorting model also lends itself to comparisons outside of NYC
itself and into the surrounding metropolitan area. This study finds the most efficient result 
is a combination of public transportation and congestion pricing \parencite[3200]{Barwick2024};
while congestion pricing improves efficiency, it disproportionately burdens lower-income drivers.
This result is highly relevant in NYC, where subway coverage is extensive but income inequality is
high. Barwick's framework suggests that CRZ revenue recycled for transit could mitigate this welfare
loss.

Together, these structural approaches extend beyond simple econometric evaluation. Brinkman emphasizes
the productivity–congestion trade-off, highlighting potential welfare ambiguity if agglomeration benefits
are lost. Barwick emphasizes equity, showing how sorting across neighborhoods mediates the distributional
effects of congestion pricing. Incorporating insights from both models allows my thesis not only to estimate
the realized impact of NYC’s CRZ, but also to evaluate its broader implications for productivity and equity.
In my opinion, the structural modeling approach is more informative and more interesting. However, 
the aforementioned lack of post-treatment data severely limits the scope of my potential research.

I aim to empirically clarify the effectiveness of these various transportation policies,
particularly congestion pricing and tolling, in reducing surface road congestiona in the New York
metro area by rigorously comparing different regional contexts. I can make preliminary guesses
to model selection at this time as model construction will necessitate a thorough statistical
process only possible with all relevant datasets.

Given our location, this thesis should have particularly important implications for us in Connecticut.
The ACS provides evidence that interstate traffic spills over off of highways and onto local
roads, so the traffic generated by the NYC metro area may certainly affect the entirety of the
state of Connecticut. Additionally, I hope, given that I find and understand some political policy
alternatives to perhaps provide a recommendation as to what might work best for our state.

\newpage
\singlespacing
\printbibliography

\end{document}
